\chapter{the theory of simple gases}
\section{an ideal gas in various quantum-mechanical ensembles}
\subsection{in a microcanonical ensemble}
\begin{itemize}
	\item 用能级 $\epsilon_i$ (不是能量本征态) 上的粒子数 $n_i$ 表示系统的状态 $\{n_i\}$, 有
	\begin{equation}
		\begin{dcases}
			\sum_i n_i = N \\
			\sum_i n_i \epsilon_i = E
		\end{dcases},
	\end{equation}
	系统的微观状态数为
	\begin{equation}
		\begin{dcases}
			W_\text{B}\{n_i\} = \prod_i \frac{(n_i + g_i - 1)!}{n_i! (g_i - 1)!} \\
			W_\text{F}\{n_i\} = \prod_i \frac{g_i!}{n_i! (g_i - n_i)!} & n_i \leq g_i
		\end{dcases},
	\end{equation}
	其中 $g_i$ 表示粒子的 $\epsilon_i$ 能级的 degeneracy.
	\begin{itemize}
		\item 用 Stirling 公式作近似,
		\begin{equation}
			\begin{dcases}
				\ln W_\text{B}\{n_i\} \approx \sum_i \Big( (n_i + g_i) \ln(n_i + g_i) - n_i \ln n_i - g_i \ln g_i \Big) \\
				\ln W_\text{F}\{n_i\} \approx \sum_i \Big( g_i \ln g_i - n_i \ln n_i - (g_i - n_i) \ln(g_i - n_i) \Big)
			\end{dcases}.
		\end{equation}
	\end{itemize}
	
	\begin{tcolorbox}[title=calculation:]
		对于 Bose--Einstein statistics, 往 $g_i$ 个格子中放 $n_i$ 个 indistinguishable 小球, 有
		\begin{equation}
			\begin{dcases}
				\frac{g_i^{n_i}}{n_i!} & \text{一个格子放多个球的情况可忽略时, 即} \ g_i \gg n_i \\
				\frac{(n_i + g_i - 1)!}{n_i! (g_i - 1)!} & \text{插板法}
			\end{dcases}
		\end{equation}
		种放法. 插板法: 球 ($n_i$ 个) 和板 ($g_i - 1$ 个) 都是 indistinguishable, 它们各自可以放在 $(n_i + g_i - 1)!$ 个位置...
		
		\noindent\rule[0.5ex]{\linewidth}{0.5pt} % horizontal line
		
		对于 Fermi--Dirac statistics, 每个格子最多一个小球, 有
		\begin{equation}
			\frac{g_i!}{n_i! (g_i - n_i)!}
		\end{equation}
		种放法.
	\end{tcolorbox}
	
	\item 最概然分布为
	\begin{equation}
		\begin{dcases}
			n_i^* = \frac{g_i}{e^{\alpha + \beta \epsilon_i} - 1} & \text{Bose--Einstein} \\
			n_i^* = \frac{g_i}{e^{\alpha + \beta \epsilon_i} + 1} & \text{Fermi--Dirac}
		\end{dcases},
	\end{equation}
	此时
	\begin{equation}
		\ln W\{n_i^*\} = \alpha N + \beta E - \frac{1}{\eta} \sum_i g_i \ln \Big( 1 - \eta e^{- \alpha - \beta \epsilon_i} \Big), \quad \text{with} \quad \eta = \begin{dcases}
			+ 1 & \text{Bose--Einstein} \\
			- 1 & \text{Fermi--Dirac} \\
			\rightarrow 0 & \text{Boltzmann}
		\end{dcases},
	\end{equation}
	其中 $\alpha = - \beta \mu$.
	
	\item 注意 $P V = T S - U + \mu N$, 得到
	\begin{equation} \label{6.1.8}
		P V = - \frac{1}{\eta} k_B T \sum_i g_i \ln(1 - \eta e^{- \alpha - \beta \epsilon_i}) \overset{\eta \rightarrow 0}{=} N k_B T.
	\end{equation}
\end{itemize}

\subsection{in a canonical ensemble}
\begin{itemize}
	\item 系统的 partition function 为
	\begin{equation}
		Z_\text{C} = \sum_{\{n_i\}} W\{n_i\} e^{- \beta E\{n_i\}},
	\end{equation}
	求和需要满足 $\sum_i n_i = N$ 的约束条件, 不方便处理, 这个约束在 grand canonical ensemble 中不存在.
\end{itemize}

\subsection{in a grand canonical ensemble}
\begin{itemize}
	\item 系统的 grand partition function 为
	\begin{equation} \label{6.1.10}
		Z_\text{GC} = \prod_i \exp \Big( - \frac{g_i}{\eta} \ln(1 - \eta e^{- \alpha - \beta \epsilon_i}) \Big) = \begin{dcases}
			\prod_i (1 - e^{- \alpha - \beta \epsilon_i})^{- g_i} & \text{Bose--Einstein} \\
			\prod_i (1 + e^{- \alpha - \beta \epsilon_i})^{g_i} & \text{Fermi--Dirac} \\
			\prod_i \exp \Big( g_i e^{- \alpha - \beta \epsilon_i} \Big) & \text{Boltzmann}
		\end{dcases},
	\end{equation}
	它们的函数图像分别为如下图 (观察到 $\epsilon$ 越大 (等价于 $\beta$ 越大), 三种统计的差别越小; $g$ 显然没有影响):
	
	\begin{figure}[H]
		\centering
		\includegraphics[scale=0.75]{figures/grand partition functions with g = 3, epsilon = 0.7.pdf}
		\caption{grand partition functions with $g = 3, \epsilon = 0.7$.}
	\end{figure}
	
	\begin{tcolorbox}[title=calculation:]
		麻烦的方法是
		\begin{equation}
			Z_\text{GC} = \sum_{\{n_i = 0\}}^{\{n_i = \text{max}\}} W\{n_i\} e^{- \beta E\{n_i\} - \alpha N\{n_i\}}.
		\end{equation}
		
		\noindent\hdashrule[0.5ex]{\linewidth}{0.5pt}{1mm} % horizontal dashed line
		
		对 Bose--Einstein statistics,
		\begin{equation} \label{6.1.12}
			Z_\text{GC} = \prod_i \sum_{n_i = 0}^\infty \frac{(n_i + g_i - 1)!}{n_i! (g_i - 1)!} (e^{- \alpha - \beta \epsilon_i})^{n_i},
		\end{equation}
		注意到 Taylor expansion
		\begin{equation}
			\frac{1}{(1 - x)^n} = \sum_{m = 0}^\infty \frac{1}{m!} \frac{(n + m - 1)!}{(n - 1)!} x^m,
		\end{equation}
		所以
		\begin{equation}
			Z_\text{GC} = \prod_i (1 - e^{- \alpha - \beta \epsilon_i})^{- g_i}.
		\end{equation}
		
		\noindent\hdashrule[0.5ex]{\linewidth}{0.5pt}{1mm} % horizontal dashed line
		
		对于 Fermi--Dirac statistics,
		\begin{equation}
			Z_\text{GC} = \prod_i \sum_{n_i = 0}^{g_i} \frac{g_i!}{n_i! (g_i - n_i)!} (e^{- \alpha - \beta \epsilon_i})^{n_i},
		\end{equation}
		注意到
		\begin{equation}
			(1 + x)^n = \sum_{m = 1}^n \frac{1}{m!} \frac{n!}{(n - m)!} x^m,
		\end{equation}
		所以
		\begin{equation}
			Z_\text{GC} = \prod_i (1 + e^{- \alpha - \beta \epsilon_i})^{g_i}.
		\end{equation}
		
		\noindent\hdashrule[0.5ex]{\linewidth}{0.5pt}{1mm} % horizontal dashed line
		
		对于 Maxwell--Boltzmann statistics,
		\begin{equation} \label{6.1.18}
			Z_\text{GC} = \prod_i \sum_{n_i = 0}^\infty \frac{g_i^{n_i}}{n_i!} (e^{- \alpha - \beta \epsilon_i})^{n_i} = \prod_i \exp \Big( g_i e^{- \alpha - \beta \epsilon_i} \Big).
		\end{equation}
		
		\noindent\rule[0.5ex]{\linewidth}{0.5pt} % horizontal line
		
		实际上, 用每个能量本征态 (而不是能级) 上的粒子数计算更方便, 只需要把 \eqref{6.1.12} \~{} \eqref{6.1.18} 中的 $g_i \mapsto 1$ 即可.
	\end{tcolorbox}
	
	\begin{itemize}
		\item 注意到这里没有用 $n_i, g_i \gg 1$ 的条件.
	\end{itemize}
	
	\item 得到
	\begin{equation}
		\begin{dcases}
			\Phi_\text{G} = - k_B T \ln Z_\text{GC} \\
			U = \sum_i \frac{g_i \epsilon_i}{e^{\alpha + \beta \epsilon_i} - \eta} \\
			N = \sum_i \frac{g_i}{e^{\alpha + \beta \epsilon_i} - \eta} \\
			\braket{n_i} = - \frac{1}{\beta} \frac{\partial}{\partial \epsilon_i} \ln Z_\text{GC} = \frac{g_i}{e^{\alpha + \beta \epsilon_i} - \eta} \\
			S = k_B \Big( \beta U + \alpha N + \ln Z_\text{GC} \Big)
		\end{dcases}.
	\end{equation}
\end{itemize}

\section{statistics of the occupation numbers}
\begin{itemize}
	\item 能级 $\epsilon_i$ 上的粒子数为
	\begin{equation}
		\braket{n_i} = \frac{g_i}{e^{\beta (\epsilon_i - \mu)} - \eta},
	\end{equation}
	其函数图像如下:
	
	\begin{figure}[H]
		\centering
		\includegraphics[scale=0.6]{figures/plot of braket{n}.pdf}
		\caption{plot of $\braket{n}$ with $g = 1$.}
	\end{figure}
	
	\item $n_i$ 的方差为
	\begin{equation} \label{6.2.2}
		\frac{\braket{(\Delta n_i)^2}}{\braket{n_i}^2} = \frac{1}{\braket{n_i}} + \frac{\eta}{g_i}.
	\end{equation}
	
	\begin{tcolorbox}[title=calculation:]
		\begin{align}
			\braket{(\Delta n_i)^2} = \braket{n_i^2} - \braket{n_i}^2 &= \frac{1}{Z_\text{GC}} \frac{1}{\beta^2} \frac{\partial^2}{\partial \epsilon_i^2} Z_\text{GC} - \frac{1}{\beta^2} \Big( \frac{\partial}{\partial \epsilon_i} \ln Z_\text{GC} \Big)^2 \notag \\
			&= \frac{1}{\beta^2} \frac{\partial^2}{\partial \epsilon_i^2} \ln Z_\text{GC} = - \frac{1}{\beta} \frac{\partial}{\partial \epsilon_i} \braket{n_i} = \cdots
		\end{align}
	\end{tcolorbox}
	
	\item 更进一步, 根据 \eqref{6.1.10}, 能量本征态 (不是能级) 占有 $n_i$ 个粒子的概率是
	\begin{equation}
		p(n_i) = \begin{dcases}
			\frac{\braket{n_i}^{n_i}}{(\braket{n_i} + 1)^{n_i + 1}} & \text{Bose--Einstein} \\
			\begin{dcases}
				\braket{n_i} & n_i = 1 \\
				1 - \braket{n_i} & n_i = 0
			\end{dcases} & \text{Fermi--Dirac} \\
			\frac{\braket{n_i}^{n_i}}{n_i!} e^{- \braket{n_i}} & \text{Boltzmann}
		\end{dcases}.
	\end{equation}
	
	\begin{tcolorbox}[title=calculation:]
		能级占有 $n_i$ 个粒子的概率是
		\begin{equation}
			p(n_i) = \begin{dcases}
				\frac{\frac{(n_i + g_i - 1)!}{n_i! (g_i - 1)!} (e^{- \alpha - \beta \epsilon_i})^{n_i}}{(1 - e^{- \alpha - \beta \epsilon_i})^{- g_i}} = \frac{(n_i + g_i - 1)!}{n_i! (g_i - 1)!} \frac{g_i^{g_i} \braket{n_i}^{n_i}}{(\braket{n_i} + g_i)^{g_i + n_i}} & \text{Bose--Einstein} \\
				\frac{\frac{g_i!}{n_i! (g_i - n_i)!} (e^{- \alpha - \beta \epsilon_i})^{n_i}}{(1 + e^{- \alpha - \beta \epsilon_i})^{g_i}} = \frac{g_i!}{n_i! (g_i - n_i)!} \frac{\braket{n_i}^{n_i} (g_i - \braket{n_i})^{g_i - n_i}}{g_i^{g_i}} & \text{Fermi--Dirac} \\
				\frac{\frac{g_i^{n_i}}{n_i!} (e^{- \alpha - \beta \epsilon_i})^{n_i}}{\exp \big( g_i e^{- \alpha - \beta \epsilon_i} \big)} = \frac{\braket{n_i}^{n_i}}{n_i!} e^{- \braket{n_i}} & \text{Boltzmann}
			\end{dcases}.
		\end{equation}
	\end{tcolorbox}
\end{itemize}

\section{kinetic considerations}
\begin{itemize}
	\item 气体的压强和 rate of effusion 分别为
	\begin{equation}
		\begin{dcases}
			P = \frac{1}{3} \frac{N}{V} \braket{|\vec{p}| |\vec{u}|} \\
			R = \frac{1}{4} \frac{N}{V} \braket{|\vec{u}|}
		\end{dcases},
	\end{equation}
	其中 $\vec{u}$ 是分子速度, 上式与统计方法无关.
	
	\begin{tcolorbox}[title=calculation:]
		都是直接从动力学计算, 令粒子速度分布为 $f(|\vec{u}|)$, ...
	\end{tcolorbox}
\end{itemize}

\section{gaseous systems composed of molecules with internal motion} \label{6.4}
\begin{itemize}
	\item 本 section 在经典统计极限下 ($n \lambda^3 \ll 1$) 计算.
	
	\item 令粒子的能级为
	\begin{equation}
		\epsilon = \underbrace{\frac{|\vec{p}|}{2 m}}_{= \epsilon_K} + \epsilon_\text{in},
	\end{equation}
	其中 $\epsilon_\text{in}$ 是 internal motion 的能量.
	
	\item 系统的 grand partition function 为
	\begin{equation}
		\frac{P V}{k_B T} = \ln Z_\text{GC} = \frac{V}{\lambda^3} z j(T),
	\end{equation}
	其中 internal partition function 为
	\begin{equation}
		j(T) = \sum_\text{internal} g_\text{in}(\epsilon_\text{in}) e^{- \beta \epsilon_\text{in}}.
	\end{equation}
	
	\begin{tcolorbox}[title=calculation:]
		\begin{align}
			\ln Z_\text{GC} &= \sum_i g_i z e^{- \beta \epsilon_i} = z \sum_{\epsilon_K} g_K(\epsilon_K) e^{- \beta \epsilon_K} \sum_\text{internal} g_\text{in}(\epsilon_\text{in}) e^{- \beta \epsilon_\text{in}} \notag \\
			&= \frac{V}{\lambda^3} z \sum_\text{internal} g_\text{in}(\epsilon_\text{in}) e^{- \beta \epsilon_\text{in}}.
		\end{align}
	\end{tcolorbox}
	
	\item 得到
	\begin{equation}
		\begin{dcases}
			N = z \frac{\partial}{\partial z} \ln Z_\text{GC} = \frac{V}{\lambda^3} z j(T) \iff \mu = k_B T (\ln n \lambda^3 - \ln j(T)) \\
			U = - \frac{\partial}{\partial \beta} \ln Z_\text{GC}(T, V, z) = N \Big( \frac{3}{2} k_B T + k_B T^2 \frac{\partial}{\partial T} \ln j(T) \Big) \\
			S = N k_B \Big( \frac{5}{2} - \beta \mu + T \frac{\partial}{\partial T} \ln j(T) \Big)
		\end{dcases}.
	\end{equation}
	
	\item 分子的 internal energy 由以下四种因素决定:
	\begin{enumerate}
		\item the electronic state,
		
		\item the state of nuclei,
		
		\item the vibrational state,
		
		\item the rotational state,
	\end{enumerate}
	严格说, these four excitation modes mutually interact, 但大多数情况下, 他们可以单独处理.
	
	\item 因此 $j(T)$ 可以写作
	\begin{equation}
		j(T) = j_\text{elec}(T) j_\text{nuc}(T) j_\text{vib}(T) j_\text{rot}(T),
	\end{equation}
	当 nuclei state 和 rotational state 的相互作用不可忽略时,
	\begin{equation}
		j(T) = j_\text{elec}(T) j_\text{nuc-rot}(T) j_\text{vib}(T).
	\end{equation}
\end{itemize}

\subsection{monatomic molecules}
\begin{itemize}
	\item 考虑温度远低于 ionization energy 的单原子气体, $T \ll \epsilon_\text{ion} / k_B \sim 10^4 \text{-} 10^5 \, \text{K}$, 此时, (几乎) 所有原子都处于 (electronic) ground state.
	
	\item nuclei state 有 spin-$S_n$, 因此 $j_\text{nuc}(T) = g_\text{nuc} = 2 S_n + 1$.
	\item 得到
	\begin{equation}
		j(T) = (2 S_n + 1) \sum_{J = |L - S|}^{L + S} (2 J + 1) e^{- \beta \epsilon_{L, S, J}},
	\end{equation}
	其中 $L, S$ 分别是电子的 orbital angular momentum 和 spin, 因为原子处于 electronic ground state, $L, S$ 是常数, (通常 $L = S = 0$), 可以简写 $\epsilon_J \equiv \epsilon_{L, S, J}$.
	\begin{itemize}
		\item 如果 $\epsilon_J \ll k_B T$, 那么
		\begin{equation}
			j_\text{elec}(T) \simeq \sum_{J = |L - S|}^{L + S} (2 J + 1) = (2 L + 1) (2 S + 1).
		\end{equation}
		
		\item 如果 $\epsilon_J \gg k_B T$, 那么
		\begin{equation}
			j_\text{elec}(T) \simeq (2 J_0 + 1) e^{- \beta \epsilon_0},
		\end{equation}
		下标 $0$ 表示 fine structure 中的 lowest energy state.
	\end{itemize}
	
	\item 系统的能量为
	\begin{equation}
		U = \begin{dcases}
			\frac{3}{2} N k_B T & \epsilon_J \ll k_B T \\
			\frac{3}{2} N k_B T + N \epsilon_0 & \epsilon_J \gg k_B T
		\end{dcases}.
	\end{equation}
	
	\item 可见, 无论在高温还是低温极限下, 气体的 specific heat 都是 $\frac{3}{2} N k_B$, 只有 entropy 和 chemical potential 受到影响.
	\begin{itemize}
		\item specific heat 在 $T \sim \Delta \epsilon_J / k_B$ 时达到极大, $\Delta \epsilon_J$ 是 separation of the fine structure levels.
	\end{itemize}
\end{itemize}

\subsection{diatomic molecules}
\begin{itemize}
	\item 同样, 考虑温度远低于 energy of dissociation, $T \ll \epsilon_\text{diss} / k_B \sim 10^4 \text{-} 10^5 \, \text{K}$, 此时, 所有分子处于 lowest electronic state.
	
	\item 考虑
	\begin{equation}
		j_\text{elec}(T) = g_0 + g_1 e^{- \beta \Delta},
	\end{equation}
	那么
	\begin{equation}
		\begin{dcases}
			U_\text{elec} = N \Delta \frac{(g_1 / g_0) e^{- \beta \Delta}}{1 + (g_1 / g_0) e^{- \beta \Delta}} \\
			(C_V)_\text{elec} = N k_B \Big( \frac{\Delta}{k_B T} \Big)^2 \frac{(g_1 / g_0) e^{- \beta \Delta}}{(1 + (g_1 / g_0) e^{- \beta \Delta})^2}
		\end{dcases},
	\end{equation}
	得到 $T$--$(C_V)_\text{elec}$ 关系如下图:
	
	\begin{figure}[H]
		\centering
		\includegraphics[scale=0.8]{figures/plot of T--(C_V)_text{elec}.pdf}
		\caption{plot of $T$--$(C_V)_\text{elec}$.}
	\end{figure}
	
	\noindent\rule[0.5ex]{\linewidth}{0.5pt} % horizontal line
	
	\item 考虑分子振动频率 $\omega$ (更高频未激发), 参考 subsection \ref{3.6.2}, harmonic oscillator 的 specific heat 为
	\begin{equation} \label{6.4.14}
		(C_V)_\text{vib} = N k_B \Big( \frac{\Theta_v}{T} \Big)^2 \frac{e^{\Theta_v / T}}{(e^{\Theta_v / T} - 1)^2}, \quad \Theta_v = \frac{\hbar \omega}{k_B},
	\end{equation}
	其图像见 figure \ref{figure 6.4 (a)}.
	
	\begin{figure}[H]
		\centering
		\begin{subfigure}{0.4\linewidth}
			\centering
			\includegraphics[scale=0.8]{figures/plot of T--(C_V)_text{vib}.pdf}
			\caption{plot of $T$--$(C_V)_\text{vib}$.}
			\label{figure 6.4 (a)}
		\end{subfigure}
		\begin{subfigure}{0.4\linewidth}
			\centering
			\includegraphics[scale=0.8]{figures/plot of T--(C_V)_text{rot}.pdf}
			\caption{plot of $T$--$(C_V)_\text{rot}$.}
			\label{figure 6.4 (b)}
		\end{subfigure}
		\caption{}
	\end{figure}
	
	\noindent\rule[0.5ex]{\linewidth}{0.5pt} % horizontal line
	
	\item nuclei states 和 rotational states 之间的相互作用在 heteronuclear molecules 中可以忽略, 但在 homonuclear molecules 中不能.
	
	\item 首先考虑 heteronuclear 的情况, rotational partition function 为
	\begin{align}
		j_\text{rot}(T) &= \sum_{l = 0}^\infty (2 l + 1) e^{- \beta l (l + 1) \hbar^2 / (2 I)} \notag \\
		&\approx \begin{dcases}
			\frac{T}{\Theta_r} + \frac{1}{3} + \frac{1}{15} \frac{\Theta_r}{T} + \frac{4}{315} \Big( \frac{\Theta_r}{T} \Big)^2 + \cdots & T \gg \Theta_r \\
			1 + 3 e^{- 2 \Theta_r / T} + 5 e^{- 6 \Theta_r / T} + \cdots & T \ll \Theta_r
		\end{dcases},
	\end{align}
	其中 $I = \frac{m_1 m_2}{(m_1 + m_2)} r_0^2$ 是分子的 moment of inertia, $\Theta_r = \frac{\hbar^2}{2 I k_B}$.
	
	\item 由此引入的 specific heat 为
	\begin{equation}
		(C_V)_\text{rot} \approx \begin{dcases}
			N k_B \Big( 1 + \frac{1}{45} \Big( \frac{\Theta_r}{T} \Big)^2 + \frac{16}{945} \Big( \frac{\Theta_r}{T} \Big)^3 \cdots \Big) & T \gg \Theta_r \\
			12 N k_B \Big( \frac{\Theta_r}{T} \Big)^2 e^{- 2 \Theta_r / T} & T \ll \Theta_r
		\end{dcases},
	\end{equation}
	其图像见 figure \ref{figure 6.4 (b)}.
	
	\noindent\hdashrule[0.5ex]{\linewidth}{0.5pt}{1mm} % horizontal dashed line
	
	\item homonuclear 的情况下...
\end{itemize}

\subsection{polyatomic molecules}
\begin{itemize}
	\item 经典近似下, rotational partition function 为
	\begin{align}
		j_\text{rot}(T) &\simeq \int_{- \infty}^\infty \frac{d\Omega dL_1 dL_2 dL_3}{h^3} e^{- \beta (L_1^2 / (2 I_1) + L_2^2 / (2 I_2) + L_3^2 / (2 I_3))} \notag \\
		&= \frac{8 \pi^2}{h^3} \prod_{i = 1, 2, 3} \sqrt{\frac{2 I_i \pi}{\beta}} = \sqrt{\pi} \prod_{i = 1, 2, 3} \sqrt{\frac{2 I_i k_B T}{\hbar^2}}.
	\end{align}
	
	\item 此时
	\begin{equation}
		(C_V)_\text{rot} = \frac{3}{2} N k_B.
	\end{equation}
	
	\item 振动与 \eqref{6.4.14} 类似, 只是多了几个振动模式.
\end{itemize}

\section{chemical equilibrium} \label{6.5}
\begin{itemize}
	\item 对于 chemical reaction
	\begin{equation}
		\nu_\text{A} \text{A} + \nu_\text{B} \text{B} \rightleftarrows \nu_\text{X} \text{X} + \nu_\text{Y} \text{Y},
	\end{equation}
	系统达到 chemical equilibrium 的条件是
	\begin{equation} \label{6.5.2}
		\nu_\text{A} \mu_\text{A} + \nu_\text{B} \mu_\text{B} = \nu_\text{X} \mu_\text{X} + \nu_\text{Y} \mu_\text{Y}.
	\end{equation}
	
	\item 如果系统是 ideal gas 或 dilute solution, 系统的 free energy 是各成分的和, 那么
	\begin{equation}
		F = N \braket{\epsilon} + N k_B T \ln(n \lambda^3) - N k_B T - N k_B T \ln j(T),
	\end{equation}
	因此
	\begin{equation}
		\mu_\text{A} = \Big( \frac{\partial F}{\partial N_\text{A}} \Big)_{T, V} = \braket{\epsilon_\text{A}} + k_B T \ln(n_\text{A} \lambda_\text{A}^3) - k_B T \ln j_\text{A}(T),
	\end{equation}
	那么 \eqref{6.5.2} 化为
	\begin{equation}
		\frac{\prod [\text{X}]^{\nu_\text{X}}}{\prod [\text{A}]^{\nu_\text{A}}} = e^{- \beta \Delta \mu^{(0)}(T)},
	\end{equation}
	其中 $[\text{A}] = n_\text{A} / n_0$, $n_0$ 可以任取, 且
	\begin{equation}
		\begin{dcases}
			\Delta \mu^{(0)} = \nu_\text{X} \mu_\text{X}^{(0)} + \nu_\text{Y} \mu_\text{Y}^{(0)} - \nu_\text{A} \mu_\text{A}^{(0)} - \nu_\text{B} \mu_\text{B}^{(0)} \\
			\mu_\text{A}^{(0)}(T) = \mu_\text{A}(n_\text{A} = n_0, T)
		\end{dcases}.
	\end{equation}
	
	\begin{tcolorbox}[title=calculation:]
		注意到
		\begin{equation}
			e^{\nu_\text{A} \beta \mu_\text{A}} = \Big( \frac{n_\text{A} \lambda_\text{A}^3}{j_\text{A}(T)} \Big)^{\nu_\text{A}} e^{\nu_\text{A} \beta \braket{\epsilon_\text{A}}},
		\end{equation}
		代入,
		\begin{equation}
			\prod_\text{A} [\text{A}]^{\nu_\text{A}} \Big( \frac{n_0 \lambda_\text{A}^3}{j_\text{A}(T)} \Big)^{\nu_\text{A}} e^{\nu_\text{A} \beta \braket{\epsilon_\text{A}}} = \prod_\text{X} [\text{X}]^{\nu_\text{X}} \Big( \frac{n_0 \lambda_\text{X}^3}{j_\text{X}(T)} \Big)^{\nu_\text{X}} e^{\nu_\text{X} \beta \braket{\epsilon_\text{X}}} \Longrightarrow \frac{\prod [\text{X}]^{\nu_\text{X}}}{\prod [\text{A}]^{\nu_\text{A}}} = \cdots
		\end{equation}
	\end{tcolorbox}
\end{itemize}
