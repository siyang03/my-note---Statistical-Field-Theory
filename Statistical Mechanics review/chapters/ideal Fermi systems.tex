\chapter{ideal Fermi systems}
\begin{itemize}
	\item $f_n(z)$ 是 Fermi-Dirac function,
	\begin{equation}
		f_n(z) \equiv \frac{1}{\Gamma(n)} \int_0^\infty \frac{x^{n - 1}}{z^{- 1} e^x + 1} dx = z - \frac{z^2}{2^n} + \frac{z^3}{3^n} - \cdots, \quad \text{and} \quad f_n'(z) = \frac{1}{z} f_{n - 1}(z),
	\end{equation}
	是单调递增函数 (当 $n > 0$), $- 1 < z < 1$, 且 $f_n(1) = (1 - 2^{1 - n}) \zeta(n)$.
	\begin{itemize}
		\item 与 Bose-Einstein statistics 不同, $f_n(z)$ 在 $z = 1$ 没有极点, 只是它的级数展开在 $z > 1$ 发散 (因为 $z = - 1$ 处有极点).
		
		\item $g_n(z) = - f_n(- z) = \mathrm{Li}_n(z)$, 称为 \href{https://en.wikipedia.org/wiki/Polylogarithm}{Polylogarithm}, 见下图:
		
		\begin{figure}[H]
			\centering
			\includegraphics[scale=0.5]{figures/plot of Li_n(z) with n=frac{3}{2}.pdf}
			\caption{plot of $\mathrm{Li}_n(z)$ with $n=\frac{3}{2}$.}
		\end{figure}
	\end{itemize}
\end{itemize}

\section{thermodynamic behavior of an ideal Fermi gas}
\begin{itemize}
	\item ideal Fermi gas 的 grand partition function 为
	\begin{equation}
		\frac{P V}{k_B T} = \ln Z_\text{GC} = N_s \Big( \frac{V}{\lambda^3} f_{5 / 2}(z) + \underbrace{\ln(1 + z)}_{\text{always negligible}} \Big),
	\end{equation}
	对 $\mu$ 的取值范围没有限制.
	
	\begin{tcolorbox}[title=calculation:]
		粒子能量的简并度见 \eqref{7.1.2}, 所以
		\begin{equation}
			\ln Z_\text{GC} = N_s \ln(1 + z) + \int_0^\infty \ln(1 + z e^{- \beta \epsilon}) g(\epsilon) d\epsilon = \cdots,
		\end{equation}
		且
		\begin{equation}
			N_0 = \frac{N_s}{z^{- 1} + 1} < N_s \Longrightarrow \ln(1 + z) = - \ln(1 - N_0 / N_s) \sim N_0 / N_s \ll N.
		\end{equation}
	\end{tcolorbox}
	
	\item 得到
	\begin{equation}
		\begin{dcases}
			N = N_s \frac{V}{\lambda^3} f_{3 / 2}(z) \\
			U = \frac{3}{2} N_s k_B T \frac{V}{\lambda^3} f_{5 / 2}(z) \\
			S = N k_B \Big( \frac{5}{2} \frac{f_{5 / 2}(z)}{f_{3 / 2}(z)} - \ln z \Big) \\
			\frac{C_V}{N k_B} = \frac{15}{4} \frac{f_{5 / 2}(z)}{f_{3 / 2}(z)} - \frac{9}{4} \frac{f_{3 / 2}(z)}{f_{1 / 2}(z)}
		\end{dcases},
	\end{equation}
	注意到 $U = \frac{3}{2} P V$ 依然成立.
	
	\item 经典极限 $n \lambda^3 \ll 1$ 对应 $z \ll 1$, 此时 $f_n(z) \simeq z$, 代入可以得到理想气体的各热力学量.
	
	\item 对 equation of state 作 virial expansion,
	\begin{equation}
		\begin{dcases}
			\frac{P V}{N k_B T} = \sum_{l = 1}^\infty (- 1)^{l - 1} a_l (n \lambda^3 / N_s)^{l - 1} \\
			\frac{C_V}{N k_B} = \frac{3}{2} \sum_{l = 1}^\infty (- 1)^{l - 1} \frac{5 - 3 l}{2} a_l (n \lambda^3 / N_s)^{l - 1}
		\end{dcases},
	\end{equation}
	其中系数 $a_l$ 见 \eqref{7.1.8}, 展开适用于 $n \lambda^3 / N_s \ll 1$.
\end{itemize}

\subsection{degenerate ideal Fermi gas}
\subsubsection{ground state of the system: $T \rightarrow 0$}
\begin{itemize}
	\item 当 $n \lambda / N_s \rightarrow \infty, z \rightarrow \infty$ 时 ($T \rightarrow 0$), mean occupation number 变成 step function,
	\begin{equation}
		\braket{n_i} = \begin{dcases}
			g_i & \epsilon < \mu_0 \\
			0 & \epsilon > \mu_0
		\end{dcases},
	\end{equation}
	其中 $\mu_0 \equiv \epsilon_F \equiv \frac{p_F^2}{2 m}$ 称为 Fermi energy, $p_F$ 称为 Fermi momentum, $k_B T_F \equiv \epsilon_F$ 称为 Fermi temperature,
	\begin{equation}
		\epsilon_F \equiv k_B T_F = \Big( 6 \pi^2 n / N_s \Big)^{\frac{2}{3}} \frac{\hbar^2}{2 m}, \quad p_F = \Big( 6 \pi^2 n / N_s \Big)^{\frac{1}{3}} \hbar.
	\end{equation}
	
	\begin{tcolorbox}[title=calculation:]
		\begin{equation}
			N = \int_0^{\mu_0} g(\epsilon) d\epsilon = N_s 2 \pi \frac{V / \lambda^3}{(\pi k_B T)^{3 / 2}} \frac{2}{3} \mu_0^{3 / 2} \Longrightarrow \mu_0 = \Big( \frac{3}{4 \pi} n \lambda^3 / N_s \Big)^{\frac{2}{3}} \pi k_B T = \cdots
		\end{equation}
	\end{tcolorbox}
	
	\item 此时, Fermi gas 处于能量基态,
	\begin{equation}
		\frac{E_0}{V} = \frac{3}{2} P_0 = N_s \frac{2 \pi}{5 m h^3} \underline{p_F^5} = \frac{3}{5} n \epsilon_F = \frac{3}{5} \Big( 6 \pi^2 / N_s \Big)^{\frac{2}{3}} \frac{\hbar^2}{2 m} \underline{n^{5 / 3}},
	\end{equation}
	可见 $P_0 \propto n^{5 / 3}$.
\end{itemize}

\subsubsection{with finite temperature}
\begin{itemize}
	\item 将 Fermi-Dirac function 关于 $\ln z = \beta \mu$ 展开, 得到
	\begin{equation} \label{8.1.10}
		\begin{dcases}
			N = N_s \frac{4 \pi}{3} V \Big( \frac{2 m}{h^2} \beta \ln z \Big)^{\frac{3}{2}} \Big( 1 + \frac{\pi^2}{8} (\ln z)^{- 2} + \cdots \Big) \Longrightarrow \mu \simeq \epsilon_F \Big( 1 - \frac{\pi^2}{12} (\beta \epsilon_F)^{- 2} \Big) \\
			U = \frac{3}{2} P V = \frac{3}{5} N \epsilon_F \Big( 1 + \frac{5 \pi^2}{12} (\beta \epsilon_F)^{- 2} + \cdots \Big) \\
			\frac{C_V}{N k_B} = \frac{\pi^2}{2} \frac{1}{\beta \epsilon_F} + \cdots \\
			S = N k_B \Big( \frac{\pi^2}{2} (\beta \epsilon_F)^{- 2} + 
			\cdots \Big)
		\end{dcases},
	\end{equation}
	可见 $T \rightarrow 0$ 时 $S \rightarrow 0$, 符合预期.
	
	\item ideal Fermi gas 的 specific heat 如下图所示:
	
	\begin{figure}[H]
		\centering
		\includegraphics[scale=0.8]{figures/specific heat of an ideal Fermi gas.pdf}
		\caption{specific heat of an ideal Fermi gas.}
	\end{figure}
\end{itemize}

\section{magnetic behavior of an ideal Fermi gas}
\begin{itemize}
	\item Maxwell-Boltzmann statistics 给出的结果见 section \ref{3.7}.
\end{itemize}

\subsection{Pauli paramagnetism}
\begin{itemize}
	\item Pauli 将 alkali metal 中的 conducting electrons 视为 highly degenerate Fermi gas.
	
	\item 电子的 Hamiltonian 为
	\begin{equation}
		H = \frac{|\vec{p}|^2}{2 m} - \mu_0 \vec{\mu}^* \cdot \vec{H},
	\end{equation}
	其中 $\vec{\mu}^* = - \mu_B \vec{\sigma}$ (见 \eqref{5.3.1}). 因此, 系统的 grand partition function 为
	\begin{equation}
		\ln Z_\text{GC} = \frac{V}{\lambda^3} \Big( f_{5 / 2}(z e^{- \beta \epsilon_{M, +}}) + f_{5 / 2}(z e^{- \beta \epsilon_{M, -}}) \Big).
	\end{equation}
	
	\begin{tcolorbox}[title=calculation:]
		将电子的能量分成 $\epsilon = \epsilon_K + \epsilon_M$, 其中 $\epsilon_M = \pm \mu_0 \mu_B H$, 所以
		\begin{align}
			\ln Z_\text{GC} &= \sum_{\epsilon_K, \epsilon_M} g_K(\epsilon_K) \ln(1 + z e^{- \beta (\epsilon_K + \epsilon_M)}) \notag \\
			&= \frac{V}{\lambda^3} \Big( f_{5 / 2}(z e^{- \beta \epsilon_{M, +}}) + f_{5 / 2}(z e^{- \beta \epsilon_{M, -}}) \Big).
		\end{align}
	\end{tcolorbox}
	
	\item 得到
	\begin{equation} \label{8.2.4}
		N_{\pm} = - \frac{1}{\beta} \frac{\partial}{\partial \epsilon_{M, \pm}} \ln Z_\text{GC} = \frac{V}{\lambda^3} f_{3 / 2}(z e^{- \beta \epsilon_{M, \pm}}) \simeq \frac{V}{6 \pi^2} \Big( \frac{2 m}{\hbar^2} (\epsilon_F - \epsilon_{M, \pm}) \Big)^{\frac{3}{2}},
	\end{equation}
	因此
	\begin{align}
		V \braket{M_z} &= N \braket{\mu^*_z} = N_+ (- \mu_B) + N_- (+ \mu_B) \notag \\
		&\simeq \frac{4 \pi^2 V \mu_B}{3} \Big( \Big( \frac{2 m}{h^2} (\epsilon_F + \mu_0 \mu_B H) \Big)^{\frac{3}{2}} - \Big( \frac{2 m}{h^2} (\epsilon_F - \mu_0 \mu_B H) \Big)^{\frac{3}{2}} \Big),
	\end{align}
	系统的 low-field magnetic susceptibility is
	\begin{equation}
		\chi_0 = \lim_{H \rightarrow 0} \frac{M_z}{H} = 4 \pi \mu_0 \mu_B^2 \Big( \frac{2 m}{h^2} \Big)^{\frac{3}{2}} \epsilon_F^{1 / 2} = \frac{3}{2} \frac{N}{V} \frac{\mu_0 \mu_B^2}{\epsilon_F}, \quad \epsilon_F(n, H = 0) = \frac{h^2}{2 m} \Big( \frac{3 N}{8 \pi V} \Big)^{\frac{2}{3}},
	\end{equation}
	其中, 用 \eqref{8.2.4} 得到 $\epsilon_F$ 的表达式, 下标 $0$ 表示 $T \rightarrow 0$ (again, highly degenerate).
	
	\begin{itemize}
		\item 对比高温极限 \eqref{3.7.18} ($g = 2, j = \frac{1}{2}$),
		\begin{equation}
			\chi_\infty = \frac{N}{V} \frac{\mu_0 \mu_B^2}{k_B T}.
		\end{equation}
	\end{itemize}
	
	\noindent\rule[0.5ex]{\linewidth}{0.5pt} % horizontal line
	
	\item 有限温度下,
	\begin{align}
		V \braket{M_z} &= N_+ (- \mu_B) + N_- (+ \mu_B) \overset{H \rightarrow 0}{=} \Big( \frac{V}{\lambda^3} f'_{3 / 2}(z) 2 z \beta \mu_0 \mu_B H \Big) \mu_B \notag \\
		&= 2 \frac{V}{\lambda^3} f_{1 / 2}(z) \frac{\mu_0 \mu_B^2}{k_B T} H = N \frac{f_{1 / 2}(z)}{f_{3 / 2}(z)} \frac{\mu_0 \mu_B^2}{k_B T} H.
	\end{align}
	\begin{itemize}
		\item 在有限 (但依然简并, $T \ll T_c$) 温度下, 参考 \eqref{8.1.10},
		\begin{equation}
			V \braket{M_z} \overset{H \rightarrow 0}{=} \Big( 2 \frac{V}{\lambda^3} f_{1 / 2}(z) \beta \mu_0 \mu_B H \Big) \mu_B \simeq V \braket{M_z}_{T = 0} \Big( 1 - \frac{\pi^2}{12} \frac{1}{(\beta \epsilon_F)^2} \Big),
		\end{equation}
		因此
		\begin{equation}
			\chi \overset{T \ll T_F}{\simeq} \chi_0 \Big( 1 - \frac{\pi^2}{12} \frac{1}{(\beta \epsilon_F)^2} \Big).
		\end{equation}
		
		\item 高温极限下, $z \ll 1 \Longrightarrow f_n(z) \simeq z - \frac{z^2}{2^n}$,
		\begin{align}
			& V \braket{M_z} \simeq \chi_\infty \Big( 1 - \Big( \frac{1}{2^{1 / 2}} - \frac{1}{2^{3 / 2}} \Big) z \Big) H \simeq \chi_\infty \Big( 1 - \frac{1}{2^{5 / 2}} n \lambda^3 \Big) H \notag \\
			\Longrightarrow & \chi \overset{T \gg T_F}{\simeq} \chi_\infty \Big( 1 - \frac{1}{2^{5 / 2}} n \lambda^3 \Big).
		\end{align}
	\end{itemize}
\end{itemize}

\subsection{Landau diamagnetism}
\begin{itemize}
	\item Landau 考虑了电子在磁场 $\vec{B} = B \hat{e}_z, \vec{A} = \frac{1}{2} \big( - B y \hat{e}_x + B x \hat{e}_y \big)$ 中的运动, 但忽略自旋, 电子的 Hamiltonian 为
	\begin{equation}
		H = \frac{1}{2 m} \Big( \vec{p} + e \vec{A} \Big)^2 = \frac{|\vec{p}|^2}{2 m} + \frac{e B}{2 m} \hat{L}_z,
	\end{equation}
	因此, 电子的能级为
	\begin{equation}
		\epsilon = \frac{\hbar^2}{2 m} \Big( \frac{\pi}{2 R_0} \Big)^2 \Big( n + \frac{1}{2} \Big)^2 + \frac{e \hbar B}{2 m} (n + n') + \frac{p_z^2}{2 m},
	\end{equation}
	其中 $n \gg 1, n' \in \mathbb{N}$.
	
	\begin{tcolorbox}[title=calculation:]
		使用 cylindrical coordinates,
		\begin{equation}
			\begin{dcases}
				r = \sqrt{x^2 + y^2} \\
				\tan \phi = \frac{y}{x} \\
				z = z
			\end{dcases} \Longrightarrow \begin{dcases}
				\nabla^2 = \frac{\partial^2}{\partial r^2} + \frac{1}{r} \frac{\partial}{\partial r} + \frac{1}{r^2} \frac{\partial^2}{\partial \phi^2} + \frac{\partial^2}{\partial z^2} \\
				\hat{L}_z = - i \hbar \frac{\partial}{\partial \phi}
			\end{dcases},
		\end{equation}
		因此, 本征值方程为
		\begin{equation}
			\Big( - \frac{\hbar^2}{2 m} \Big( \frac{\partial^2}{\partial r^2} + \frac{1}{r} \frac{\partial}{\partial r} + \frac{1}{r^2} \frac{\partial^2}{\partial \phi^2} + \frac{\partial^2}{\partial z^2} \Big) - i \frac{e \hbar B}{2 m} \frac{\partial}{\partial \phi} - \epsilon \Big) \psi(r, \phi, z) = 0,
		\end{equation}
		分离变量
		\begin{equation}
			\psi(r, \phi, z) = e^{i p_z z / \hbar} e^{i j \phi} R_{j, p_z}(r) \Longrightarrow r^2 R'' + r R' - (j^2 - \kappa_j^2 r^2) R = 0,
		\end{equation}
		其中 $\frac{\hbar^2 \kappa_j^2}{2 m} = \epsilon - \frac{e \hbar B}{2 m} j - \frac{p_z^2}{2 m}, j \in \mathbb{N}$, 令 $\rho = \kappa_j r$, 那么
		\begin{equation}
			\rho^2 R'' + \rho R' + (\rho^2 - j^2) R = 0,
		\end{equation}
		得到
		\begin{equation}
			R_j(\rho) = a J_j(\rho) + b Y_j(\rho),
		\end{equation}
		which are the \href{https://en.wikipedia.org/wiki/Bessel_function}{Bessel functions}. 注意到, 虽然
		\begin{equation}
			J_{- j}(\rho) = (- 1)^j J_j(\rho), \quad Y_{- j}(\rho) = (- 1)^j Y_j(\rho),
		\end{equation}
		但 $e^{\pm i j \phi}$ 分别对应不同的态. 在 $\rho \gg 1$ 时, 其渐近形式为
		\begin{equation}
			\begin{dcases}
				J_j(\rho) \simeq \sqrt{\frac{2}{\pi \rho}} \cos \Big( \rho - \frac{j \pi}{2} - \frac{\pi}{4} \Big) \\
				Y_j(\rho) \simeq \sqrt{\frac{2}{\pi \rho}} \sin \Big( \rho - \frac{j \pi}{2} - \frac{\pi}{4} \Big)
			\end{dcases},
		\end{equation}
		边界条件设置为 $r = R_0$, 令 $\kappa_j R_0 = \rho_0 \gg 1$, 那么
		\begin{equation}
			\begin{dcases}
				R_j(\rho) = J_j(\rho) & j = \underbrace{\frac{1}{2} + \frac{2 \rho_0}{\pi}}_{= n} + 2 n_J \\
				R_j(\rho) = Y_j(\rho) & j = \underbrace{- \frac{1}{2} + \frac{2 \rho_0}{\pi}}_{= n - 1} + 2 n_Y
			\end{dcases},
		\end{equation}
		注意到 $j \in \mathbb{N}$, 因此
		\begin{equation}
			\kappa_j R_0 = \rho_0 = \frac{\pi}{2} \Big( n + \frac{1}{2} \Big) \gg 1,
		\end{equation}
		得到系统的能级
		\begin{equation}
			\epsilon = \frac{\hbar^2}{2 m} \Big( \frac{\pi}{2 R_0} \Big)^2 \Big( n + \frac{1}{2} \Big)^2 + \frac{e \hbar B}{2 m} \underbrace{(n + n')}_{= j} + \frac{p_z^2}{2 m}, \quad n' = \begin{dcases}
				2 n_J \\
				2 n_Y - 1
			\end{dcases},
		\end{equation}
		因此每个 $(n, n', p_z)$ 对应电子的一个能量本征态.
	\end{tcolorbox}
	
	\item 那么, 电子的能级简并度和系统的 grand partition function 为
	\begin{equation}
		\begin{dcases}
			g(\epsilon) = 64 \pi V \frac{m^{5 / 2}}{e h^4 B} \epsilon^{1 / 2} \\
			\ln Z_\text{GC} = 8 \sqrt{2} \frac{V}{\lambda^3} \frac{m}{e h B} f_{5 / 2}(z)
		\end{dcases}.
	\end{equation}
	
	\begin{tcolorbox}[title=calculation:]
		注意 $\frac{p_z^2}{2 m} = \frac{h^2}{8 m L^2} n_z^2$,
		\begin{align}
			\frac{1}{2} g(\epsilon) &= \int_0^\infty dj \int_0^\infty dn_z \, \delta(\epsilon - \epsilon(j, n_z)) = \int_0^\infty dj \, \frac{1}{\frac{h^2}{4 m L^2} n_z} \Big|_{\epsilon, j} \theta(\epsilon - \epsilon(j, 0)) \notag \\
			&= \frac{1}{2} \Big( \frac{8 m L^2}{h^2} \Big)^{1 / 2} \sqrt{\frac{m}{e \hbar B}} \int_0^{\frac{m}{e \hbar B} \epsilon - \frac{1}{2}} \frac{dj}{\sqrt{\big( \frac{m}{e \hbar B} \epsilon - \frac{1}{2} \big) - j}} \notag \\
			&= \Big( \frac{8 m L^2}{h^2} \Big)^{1 / 2} \sqrt{\frac{m}{e \hbar B}} \sqrt{\frac{m}{e \hbar B} \epsilon - \frac{1}{2}} \simeq \cdots,
		\end{align}
		因此, 系统的 grand partition function 为
		\begin{align}
			\ln Z_\text{GC} &=  64 \pi V \frac{m^{5 / 2}}{e h^4 B} \int_0^\infty \ln(1 + z^{- \beta \epsilon}) \epsilon^{1 / 2} d\epsilon \notag \\
			&= 64 \pi V \frac{m^{5 / 2}}{e h^4 B} \frac{\sqrt{\pi}}{2} \beta^{- 3 / 2} f_{5 / 2}(z) = \cdots
		\end{align}
	\end{tcolorbox}
\end{itemize}

\section{the electron gas in metals}

\section{ultracold atomic Fermi gases}

\section{statistical equilibrium of white dwarf stars}
