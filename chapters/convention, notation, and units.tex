\chapter*{convention, notation, and units}
\addcontentsline{toc}{chapter}{convention, notation, and units}
\begin{itemize}
	\item 使用 Planck units, 此时 $G, \hbar, c, k_B = 1$, 因此:
	
	\begin{center}
		\begin{tabularx}{\linewidth}{XX}
			\toprule 
			name/dimension & expression/value \\
			\midrule 
			Planck length ($L$) & $l_P = \sqrt{\frac{\hbar G}{c^3}} = 1.616 \times 10^{- 35} \, \text{m}$ \\
			Planck time ($T$) & $t_P = \frac{l_P}{c} = 5.391 \times 10^{- 44} \, \text{s}$ \\
			Planck mass ($M$) & $m_P = \sqrt{\frac{\hbar c}{G}} = 2.176 \times 10^{- 8} \, \text{kg} \simeq 10^{19} \, \text{GeV}$ \\
			Planck temperature ($\Theta$) & $T_P = \sqrt{\frac{\hbar c^5}{G k_B^2}} = 1.417 \times 10^{32} \, \text{K}$ \\
			\bottomrule
		\end{tabularx}
	\end{center}
	
	\item 时空维度用 $d = D + 1$ 表示.
	
	\noindent\rule[0.5ex]{\linewidth}{0.5pt} % horizontal line
	
	\item 下面是 \textit{Statistical Field Theory}, David Tong, 中引言的一部分.
	
	\begin{tcolorbox}
		... This phenomenon is known as \textit{universality}.
		
		All of this makes phase transitions interesting. They involve violence, universal truths
		and competition between rival states. The story of phase transitions is, quite literally,
		the song of fire and ice.
		
		...
		
		... This leads us to a paradigm which now underlies huge swathes of physics, far removed from its humble origin of a pot on a stove. This paradigm revolves around two deep facts about the Universe we inhabit: \textbf{Nature is organised by symmetry. And Nature is organised by scale.}
	\end{tcolorbox}
	
	其中 symmetry 指 Landau's approach to phase transitions, scale 是指 renormalization group.
\end{itemize}
